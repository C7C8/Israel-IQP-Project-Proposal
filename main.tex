\documentclass{article}                         % Make it an article
\usepackage{indentfirst}                       % All paragraphs indented
\usepackage{setspace}                           % For 1.5 spacing
\usepackage[hidelinks]{hyperref}      % Make table of contents clickable
\usepackage[margin=1.0in]{geometry} % 1 inch margin
\usepackage[english]{babel}                     % For "fancy" quotes
\usepackage[autostyle, english=american]{csquotes} % Same
\usepackage{apacite}                            % Use APA citations
\usepackage{etoolbox}                           % Misc
\MakeOuterQuote{"}                              % For fancy quotes
% Turn off URLs in citations. Don't delete this in case we need it again.
% https://tex.stackexchange.com/questions/425392/how-remove-links-from-apacite
%\renewenvironment{APACrefURL}[1][]{}{}
%\AtBeginEnvironment{APACrefURL}{\renewcommand{\url}[1]{}}

\title{Israel IQP Project Proposal}
\author{Evan Goldstein, Valeria Kopper, Christopher Myers, Zachary Zlotnick}
\date{November 2018}

\begin{document}
\pagenumbering{roman}
\maketitle

\renewcommand\abstractname{Summary} % "Abstract" -> "Summary", the header lies
\begin{abstract}
This short section at the top should be an executive summary of the rest of the document, something like an abstract. It should be very short (one paragraph at most) and extremely concise, covering the proposal at a very high level. It may be appropriate to put sponsor acknowledgments here if they're needed. \textbf{\textit{This section might not be needed! Awaiting feedback from advisor.}}
\end{abstract}

\tableofcontents
\newpage
\pagenumbering{arabic}

% Actual content begins here!
\doublespacing

\section{Introduction}
\textit{Not part of this submission.}

\newpage
\section{Background}
\subsection{Electric vehicles, the smart grid, \& vehicle-to-grid charging}

Electric vehicles (EVs) are an energy efficient alternative to traditional combustion-powered vehicles, relying on built-in batteries and motors to store energy and move the vehicle. Traditional vehicles produce carbon dioxide (CO$^2$) emissions from burning gasoline, but convert only 10-15\% of the energy in the gasoline into energy to move the vehicle. In comparison, a hybrid vehicle --- combining combustion and electric technology --- can yield 30-40\% efficiency \cite{Zhu2015DistributedGrid}, with the remaining energy similarly wasted. Pure EVs operate on much simpler principles, saving energy and reducing carbon emissions.

Modern EVs are currently limited in both battery capacity and range, presenting a usage-awareness challenge to consumers who wish to switch. On the low end of the scale, cars can have as little as 18.8 kWh in energy storage with a range of only 100 miles, while higher end models can store up to 53 kWh for a range of 220 miles \cite{Ustun2015ImpactSystems}. For reference, a typical household might use 10,000 kWh or more in a year. Depending on travel habits and local availability of charging stations, this may limit adoption. EVs also increasingly require dedicated, high-power charging systems, namely "fast chargers" requiring up to 70 kW (reference: A normal laptop might draw 60-70 W while charging) to charge one vehicle \cite{Ustun2015ImpactSystems}. If many of these vehicles were in use in one area, the load on the local electric grid during peak hours could be enormous.

To help satisfy energy requirements and reduce carbon emissions, a number of renewable energy options are available to Eilat and Israel as a whole, most notably solar power. Solar power requires large amounts of exposure to the sun (which Israel has plenty of) and huge swathes of land to generate power effectively (modern photovoltaic solar panels are only 10-17\% efficient \cite{Zhu2015DistributedGrid}), but solar panels can also be mounted as rooftop installations. In a best-case scenario with solar panels on every roof, up to 32\% of Israel's energy requirements could be satisfied by solar, while in a more realistic scenario of solar panels on only large roofs ($>$800m$^2$) up to 7\% could be satisfied \cite{Vardimon2011AssessmentIsrael}, and this is only for rooftop deployments. Far more power could be generated by dedicated solar farms, which Israelis also support \cite{Dipersio2018PhotovoltaicAcceptance}. Unfortunately, the volatile nature of solar energy (dependent on weather and time of day) can cause problems with satisfying power demand \cite{Lu2015IntroductionPEVs} and the deployment of dedicated solar farms would likely require significant upgrades to the existing power grid \cite{Vardimon2011AssessmentIsrael} in both capacity and technology. Combined with the high energy demands imposed by a fleet of electric vehicles, a desperately-needed increase in solar power and 90\% EVs (as per Eilat's goal) could reasonably be expected to have a devastating effect on the existing grid.

To address both these problems and other current electric demand problems, such as how to accommodate the rise in demand towards the evenings (peak time) the concept of a \textit{smart grid} has been proposed. Briefly, a smart grid is a power grid where power production, storage, and transmission is regulated intelligently by computers in an on-demand fashion. These usually involve "microgrids" that contain a number of "loads" (something that draws power, like a house) and local sources of power, such as batteries or rooftop solar panels \cite{Lu2015IntroductionPEVs}. One proposed use of smart grids is to connect EVs when they're not in use, charging their batteries when power is more readily available and discharging during peak time \cite{Mahmud2015PowerEV} or times of exceptional demand. Ideally, this would help relieve stress on power generators and reduce waste from long-range power transmission \cite{Zhu2015DistributedGrid}. Together the connected EVs would serve as a distributed battery; these concepts are the core of vehicle-to-grid (V2G) charging.

Unfortunately, the technology needed has not been fully investigated and the transition phase may be difficult. Consumers may be reluctant to switch to EVs due to their range or availability of charging stations \cite{Zhu2015DistributedGrid}. This is a cyclical problem since fewer consumers transitioning implies fewer charging stations, since there would be little incentive to build many stations for a small fleet of EVs. Universal standards for charging cars and the complicated control circuity involved in two-way power flow may also need to be developed \cite{Ustun2015ImpactSystems, Yeshayahou2011IsraelCharging}, not to mention the actual infrastructure and control systems underpinning smart grids. A smart grid and V2G charging in Eilat may also need adapting to their local environment, depending on factors such as Israeli travel habits (a car can only serve as a battery for the grid if it's not in use, for instance), frequency of use, or Eilat citizens' individual willingness and financial readiness to switch to EVs.

\subsection{Existing economy}
Israel is not traditionally considered a strong player within the global automotive industry; however, they are proving to be a country on the rise in this regard. As a nation, Israel is known for the environment it has provided for technology start-ups to thrive. That in mind, many global brands are deciding to invest in the "high-end automotive technology sector." Some of these global brands include Intel, looking to apply its software automation in Jerusalem; Volkswagen AG, trying to popularize the ride hailing service called "Gett,"; even BMW AG, who made a large investment in a transit app called "Moovit" \cite{Coutinho2018IsraelIndustry}. 

With regard to the traditional facets of the automotive industry, the Israeli market also has companies specializing in Tier 1, 2, and 3 supply (x, y, and z, respectively)*****, which manufacture and assemble components for final vehicle assembly. As of 2015, Israel has over 60 companies working in Tier 1 aftermarket supply creating spare parts and accessories such as batteries, ventilation, gears and valves. Israel also has over 50 companies that supply sub-assembly components to original equipment manufacturers (OEMs) \cite{MinistryofEconomyandIndustryStateofIsraelTheIsrael}. These companies may begin to shift over time as Israel seeks alternatives to traditional, inefficient cars, and moves toward lighter, smarter, and technologically savvy electric vehicles.

Keeping in mind that Israel is looking to move toward smart cars and efficient vehicles, there are a number of notable trends occurring in the world right now. One shift being made is the reduction of vehicle weight by using alternative materials, given that a car that is 10\% lighter would use 6-8\% less fuel \cite{MinistryofEconomyandIndustryStateofIsraelTheIsrael}. Other popular trends lie in the information technology sector, enabling vehicles to communicate with other vehicles and become more automated. These technologies include driver assistance systems, automated and smart transportation, and information security. All of these technologies aim to reduce the inherent error that human drivers have, decreasing commute times, reducing the impact of potential crashes, and keeping personal information out of the hands of hackers \cite{MinistryofEconomyandIndustryStateofIsraelTheIsrael}. Directly related to this increase in technology within the automotive sector, there exist some ethical considerations that must be taken into account. While the idea of more efficient vehicles would not receive much opposition, technology which takes control from the driver is heavily debated. Questions regarding the culpability of the driver in certain situations come into play, as well as driver rights, manufacturer liability, and implications for future generations. 

Responding to this fundamental shift in the way the world is viewing transportation, Israel has responded with a few ideas and companies of their own. For example, the trending navigation app called "Waze" was based in Israel before they were bought out by Google in 2013. Waze encourages communication between drivers via a mobile app (not while driving obviously) where users can report traffic jams, vehicles stopped on the shoulder, hazards, and even police officers. Another Israeli company, "Cellint," monitors all active cell phones and detects their exact location on the roads every 30 seconds, collecting data to use for traffic management, research, and time-to-destination analysis. Other Israeli companies such as "Argus" aim to address the issue of information security, ensuring that data can be shared between vehicles seamlessly without intervention from hackers \cite{MinistryofEconomyandIndustryStateofIsraelTheIsrael}. The list of Israel-based companies and their various applications goes on, however their goals all relate to creating safer, smoother and more efficient journey for drivers.

Overall, the Israeli automotive industry is a developing one. This is not to say that it is underdeveloped, but that it has focused its lens on emerging technologies and brand new global needs. International companies have focused their attention to Israel as well in search of new viable automobile-related technologies. In addition to the cutting-edge technologies that have sprouted in Israel, they also have made efforts to vertically integrate many of them - ensuring that there is support from initial design to final production \cite{MinistryofEconomyandIndustryStateofIsraelTheIsrael}. Lastly, Israel's location enables them to get the best of both commercial worlds; they can easily trade with Asia and Europe.

\subsection{Transportation (Evan)}
Eilat is a small city with a population of 47,800. Within the city center, you can walk almost everywhere. There is one bus company, Egged. \cite{TransportationEilat} Get Taxi, an Uber-like service, is also available. 

A March 2018 Bank of Israel report found that commuting times in the country have increased by 30\% in the last decade. This is likely because private and commercial vehicles account for how 69\% of Israelis get to work, nationwide. Following that, 18\% take a bus. Tel Aviv and Haifa are very close to the national numbers, but in Jerusalem, the statistic changes drastically, to 54\% car and 32\% bus \cite{Dori2018IsraeliRoads}. By comparison, in European cities like Barcelona and Brussels public transportation accounted for 40\% of all commuting and in Paris, 70\%. The report found that those who do take the bus to work do so not because it is convenient or efficient, but because they cannot afford to own a car, and therefore have no choice. The report came after the International Monetary Fund said Israel’s transportation infrastructure was insufficient and was threatening to be a drag on economic growth. 

Israeli car ownership has increased rapidly in the past 20 years. In 1995, car ownership was climbing at six to seven percent each year, among the fastest in the world at the time \cite{Slater1995IsraelCulture}. A 2012 report from the Israel Tax Authority showed that car owners in the center and north of the country drove the least—around 14,000 kilometers a year on average between the Tel Aviv, Jerusalem and Haifa areas—while car owners in far-north Metula and environs drove the most, at 22,000 kilometers a year. Eilat Car owners, who are the farthest from the country’s center, drove the second most at 19,000 kilometers a year \cite{Schmil2012WhatHow}. The report states that between 1995 and 2011, tax revenue from cars more than doubled. Additionally, whether due to increasing green-awareness or Israel’s Green Tax Reform, cars on Israel's roads are becoming less damaging to the environment\cite{Schmil2012WhatHow}.

Luxury car ownership has also increased rapidly, doubling in recent years \cite{Weinglass2018Rev-upIsrael}. Economist Yaron Zalicha told the Times of Israel: “The low interest rates of recent years allowed car purchasers to borrow more money and incentivized consumption over saving. …In addition, the government recently made it easier for people to import cars individually, which incentivizes the import of expensive cars” \cite{Weinglass2018Rev-upIsrael}. Some see the rise of luxury cars as being a direct result of a growing economy. As in many places, there is a stigma against owning a luxury car. It is a common stereotype in Israel that the owners of super-luxury cars are criminals or tax evaders. This is partly because until a new goes into effect in 2019 it is legal in Israel to purchase apartments and cars using cash.

Israel does not have a particularly thriving car culture scene. This is largely limited by the government. Speed limits are low, and every single modification to a vehicle, even changing the wheel size, requires special permission. New car purchases are taxted at 83\%, making the purchase if a car very expensive.

\subsection{Oil imports \& Israeli independence}
	Over the course of recent modern history the entire modern world has developed a deep reliance on oil, and the State of Israel is no exception to this consumerist world culture that largely revolves around oil-dependent products. Oil is necessary for society to function as many daily activities require oil, and while many countries, organizations and individuals attempt to decrease this strong dependence, such as through the investment in renewable energy production and research within this scope, everyone is still inescapably and intrinsically tied to oil. For some countries this means large oil production and exports, which often times becomes the driving force of their country's economy. For other countries, this implies being tied and subject to the countries responsible for sourcing their oil imports. Among the former category fall many middle eastern nations; however, Israel is not among them. Quite the opposite, given that Israel produces incredibly small quantities of oil, as little as only "a couple thousand barrels of oil a day" \cite{Engber2006WhereOil}, forcing the nation to heavily rely on oil imports. Such a small oil production does not allow Israel to sustain more than 1\% of its oil needs, which means it imports over 99\% of its oil \cite{Engber2006WhereOil}. Many of these imports originate from its neighboring countries; countries with whom Israel has a long and complicated history of tense relationships. This generates an undesired and convoluted vulnerability for Israel.
    
    Israel's oil import distribution among exporting countries is undisclosed because, as Israel's minister of national infrastructures (in 2004) stated, "Israel's situation is complicated. We don't have diplomatic relations with most of the countries from which we import oil." \cite{Engber2006WhereOil} Over the years Israel has imported oil from a wide range of countries, including but not limited to, Angola, Colombia, Mexico, Egypt, and Norway \cite{Engber2006WhereOil}. More recently these imports have shifted towards a large majority originating from Russia and Kazakhstan \cite{Engber2006WhereOil}. 
	
    Since its establishment, the State of Israel has continuously sought oil resources closer to home in order to increase stability and safety as well as decreasing import costs. This notion rose to particular prominence in 1973 when Israel went through an oil crisis, which led to large imports from Iran through a pipe referred to as the "TIPline" \cite{Engber2006WhereOil}. This pipe allowed for Iranian oil extraction from the Red Sea; however, this only lasted until the Shah was overthrown in 1979 \cite{Engber2006WhereOil}. This led to Israel seizing control of oil sources in the Sinai Fields in Egypt after the Six-Day War took place \cite{Engber2006WhereOil}.  Afterwards Israel and Egypt have maintained a friendly relationship within the oil context; however, Israel's demand for oil has grown, leaving Egypt as a minor source of import. Its overall growth and increase of oil demand has further fostered Israel's need to decrease their dependence on other nations for almost the entirety of their oil resources. 

	Ideally, Israel can decrease its dependence on other nations by decreasing its oil dependence. Decreasing energy consumption should be a global goal, but at this given point in time cannot be not be the sole focus of efforts going into decreasing oil consumption specifically. On the other hand, focusing on the development of energy consumption in future years progressing towards the use of renewable energy instead of oil consumption would yield great benefits in terms of environmental, social and political aspects. This can be achieved by focusing on renewable energy production and shifting towards electric machines, such as electric vehicles, versus oil-run machines. Israel has the opportunity to expanding upon its untapped potential for renewable energy production, especially solar polar, which would grant Israel an increased independence from its oil imports and undesired ties to neighboring countries. 

\subsection{Summary}

\newpage
\section{Methods}
\textit{Not part of this submission.}

% Bibliography -- automatically managed in APA style, on a new page
\newpage
\bibliographystyle{apacite}
\bibliography{refs}

\end{document}

\documentclass{report}                         % Make it a report 
\usepackage{indentfirst}                       % All paragraphs indented
\usepackage{setspace}                           % For 1.5 spacing
\usepackage[hidelinks]{hyperref}      % Make table of contents clickable
\usepackage[margin=1in]{geometry} % 1 inch margin
\usepackage[english]{babel}                     % For "fancy" quotes
\usepackage[autostyle, english=american]{csquotes} % Same
\usepackage{apacite}                            % Use APA citations
\usepackage{etoolbox}                           % Misc
\usepackage{titlesec}
\titleformat{\chapter}[display]{\normalfont\bfseries}{}{0pt}{\Huge} % Make headers reasonable again 
\titlespacing*{\chapter}{0pt}{-30pt}{0pt}
\titlespacing*{\section}{0pt}{0pt}{0pt}
\titlespacing*{\subsection}{0pt}{0pt}{-5pt}
\MakeOuterQuote{"}                              % For fancy quotes


\title{Israel IQP Project Proposal}
\author{Evan Goldstein, Valeria Kopper, Christopher Myers, Zachary Zlotnick}
\date{November 2018}

\begin{document}
\pagenumbering{roman}
\maketitle

\renewcommand\abstractname{Summary} % "Abstract" -> "Summary", the header lies

\tableofcontents
\newpage
\pagenumbering{arabic}

% Actual content begins here!
\doublespacing

\chapter{Introduction}

Modern Israel relies on conventional (gasoline and diesel-powered) vehicles for transportation. With nearly 70\% of Israeli citizens driving to work and another 18\% taking buses, commute times have increased 30\% over the past decade \cite{Dori2018IsraeliRoads}, increasing oil consumption and worsening the dependence on it. The environmental impacts of oil usage are well known, but in Israel's case oil usage is more problematic. Israel produces only a few thousand barrels of oil per day and therefore imports 99\% of its oil \cite{Engber2006WhereOil}. In modern times this oil comes from Russia, Egypt, and other nearby Arabic nations -- considering Israel's troubled past with its neighbors and several oil crisis incidents and treaties \cite{Engber2006WhereOil}, Israel has serious motivation to decrease its reliance on oil in favor of more readily available sources of power.

To that end, Eilat is investigating the possibility of heavily promoting electric vehicles (EVs), along with related infrastructure and possibly a vehicle-to-grid (V2G) charging system. Briefly, electric vehicles are vehicles such as cars, buses, or even trucks that rely on traction power generated from an electric motor instead of an engine, and powered by a very high capacity lithium-ion battery instead of gasoline. Electric cars are up to 4 times more efficient than traditional internal combustion vehicles, and have been successfully demonstrated in cars such as the Tesla Roadster, with a range of up to 393 km \cite{Friel2010ManagementVehicles}. While this particular vehicle might be a higher end car, it's a good example of the viability of all-electric transportation.

Much work has already been done on investigating the benefits and drawbacks of EVs from economic, environmental, and engineering perspectives. As previously stated electric cars are well-known to be more energy efficient, but their overall impact varies with the source of the electricity. That is, if the energy is sourced from renewable sources (e.g. solar, the current best option for Eilat), all-electric is advantageous, while if the energy is sourced from fossil fuels like oil, it's more efficient to generate power on-board, such as in a hybrid electric vehicle \cite{Dincer2010EconomicOptions}, and the same goes for environmental effects. V2G charging is the idea of allowing two-way power flow between a car and the electric grid, allowing the car battery to supply power to the grid when needed, while charging when power is more readily available \cite{Mahmud2015PowerEV}. Unfortunately, the complicated equipment underpinning the idea are still under development \cite{2018Nuvve...} and have not been fully demonstrated in a complicated urban environment under real-world conditions.

The problem with applying efforts under development elsewhere to Eilat's situation is that they do not take into account Eilat's demographics and power sources. Eilat aims to transition to electric vehicles \textit{and} increase solar usage, which can interfere with part of the benefits of V2G --- charging would take place at times of decreased load, usually at night when solar panels coincidentally cannot generate power. Without adequate power management, it's entirely possible that an EV owner might wake up one morning only to find that their car has discharged completely overnight. Furthermore, without an understanding of what motivates Israelis to switch to electric cars (the average consumer is unwilling to pay extra in the name of environmental friendliness \cite{Mock2010MarketVehicles}) it will be difficult to determine how the city should properly incentivize its citizens. The applicability of V2G charging to Eilat has also not been investigated, with little to no data available on what sort of charging station frequency or range would be acceptable to those who want to switch. Finally, if Eilat wishes to switch both to solar panels and electric cars, serious upgrades to their power would be required \cite{Vardimon2011AssessmentIsrael}, but whether this is feasible or not is undetermined.

The goal of this project is to assess the technological viability and requirements of Eilat's goal to have 90\% electric vehicles by 2040, and develop a roadmap or guide for its implementation in order to aid the city of Eilat in switching to electric vehicles.
We aim to collect data on social acceptance of EVs in Eilat, investigate the current state of Eilat's power grid, and analyze social acceptance \& technological data to determine whether implementation by 2040 is realistic. Although adapting the technology to Eilat's situation and convincing citizens to convert to EVs may prove challenging, it opens unique opportunities. Eilat's effort can prove V2G's potential, demonstrate a working EV-powered city, and act as a role model for other cities looking to switch.

\newpage
\chapter{Background}
\textit{Need an introduction here!}

\section{Transportation}
Eilat is a relatively small city with a population of 47,800. Within the city center, you can walk almost everywhere. There is only one bus company, Egged. \cite{TransportationEilat} "Get Taxi", an Uber-like service, is also available. 

In March 2018, a report published by the Bank of Israel found that commute times in the country have increased by 30\% in the last decade. This is likely because private and commercial vehicles account for how 69\% of Israelis get to work nationwide. Following that, 18\% take a bus. Tel Aviv and Haifa are very close to the national numbers, but in Jerusalem, the statistic changes drastically, to 54\% car and 32\% bus \cite{Dori2018IsraeliRoads}. By comparison, in European cities like Barcelona and Brussels public transportation accounted for 40\% of all commuting and in Paris, 70\%. The report found that those who do take the bus to work do so not because it is convenient or efficient, but because they cannot afford to own a car, and therefore have no choice. The report came after the International Monetary Fund said Israel’s transportation infrastructure was insufficient and was threatening to be a drag on economic growth. 

Israeli car ownership has increased rapidly in the past 20 years. In 1995, car ownership was climbing at 6-7\% per year --- among the fastest in the world at the time \cite{Slater1995IsraelCulture}. A 2012 report from the Israel Tax Authority showed that car owners in the center and north of the country drove the least --- around 14,000 kilometers a year on average between the Tel Aviv, Jerusalem and Haifa areas --- while car owners in far-north Metula area drove the most, at 22,000 kilometers a year. Eilat car owners, who are the farthest from the country’s center, drove the second most at 19,000 kilometers a year \cite{Schmil2012WhatHow}. The report states that between 1995 and 2011, tax revenue from cars more than doubled. Additionally, whether due to increasing green-awareness or Israel’s Green Tax Reform, cars on Israel's roads are becoming less damaging to the environment\cite{Schmil2012WhatHow}.

% Luxury car ownership has also increased rapidly, doubling in recent years \cite{Weinglass2018Rev-upIsrael}. Economist Yaron Zalicha told the Times of Israel: “The low interest rates of recent years allowed car purchasers to borrow more money and incentivized consumption over saving. …In addition, the government recently made it easier for people to import cars individually, which incentivizes the import of expensive cars” \cite{Weinglass2018Rev-upIsrael}. Some see the rise of luxury cars as being a direct result of a growing economy. As in many places, there is a stigma against owning a luxury car. It is a common stereotype in Israel that the owners of luxury cars are criminals or tax evaders. This is partly because until a new law goes into effect in 2019 it is legal in Israel to purchase apartments and cars using cash.

Israel does not have a particularly thriving car culture scene. This is largely limited by the government. Speed limits are low, and every single modification to a vehicle, even changing the wheel size, requires special permission. New car purchases are taxed at 83\%, making the purchase of a car very expensive.

\section{Electric vehicles \& solar power}
\subsection{Electric vehicles}
Electric vehicles (EVs) are an energy efficient alternative to traditional internal combustion engine vehicles (ICEVs), relying on built in lithium-ion batteries to store power and electric motors to generate traction to move the vehicle. ICEVs produce carbon dioxide (CO$^2$, a known greenhouse gas) emissions from burning gasoline, but convert 10-15\% of the chemical energy within into traction that moves the car. In comparison, a hybrid vehicle --- combining both combustion and EV technology --- can yield 30-40\% efficiency \cite{Zhu2015DistributedGrid}. For a more concrete example, gasoline has a known energy density of 9.5 kWh/liter \cite{EngineeringFactors}, and a 2018 Honda civic sedan has 46.9 liter tank with a fuel efficiency of 14.9 km/l \cite{20182018Information}. This means it can store 445.5 kWh with a range of 698 km. In comparison, a Tesla Roadster has a max range of 393 km but a battery capacity of only 60 kWh \cite{Friel2010ManagementVehicles}, making the Roadster over 4 times more energy efficient despite its lower range.

As can be seen, EVs are currently limited in both battery capacity and range, presenting a usage-awareness challenge to consumers who wish to switch. On the low end of the scale, cars can have as little as 18.8 kWh batteries with a range of only 160 km, while higher end models like the Tesla Roadster from above have 60 kWh batteries and a range of 393 km \cite{20182018Information}. For reference, Israelis use 6,893 kWh in electric power per capita as of 2018 \cite{2018Key2018}. Depending on travel habits and local availability of charging stations. The key factor in EV range is battery capacity, which unfortunately remains expensive, costing up to 1,000 USD/kWh in 2010 \cite{Mock2010MarketVehicles}, although future projections estimate that \$200-300/kWh is a realistic goal and mass production is likely to further reduce expense. Range is a known concern of drivers, who on their own have little incentive to switch to EVs \cite{Mock2010MarketVehicles}, so the best possible range and availability of charging stations is essential.

\subsection{Charging \& power requirements}
EVs have significant power requirements while charging that require special accommodations. Chargers are broken down into levels 1-4, with increasing charge rate and infrastructure requirements as you progress along the scale. Modes 1 and 2 are most common currently and consist of simply plugging the EV into AC power (mode 2 has a special in-line control box for added safety) \cite{Bossche2010ElectricInfrastructure}, using hardware built into the car to convert AC to DC power. These chargers are considered less safe but are ideal for opportunity charging (plugging in your car whenever parked for an extra top-up) and don't require special infrastructure, although they can take up to 18 hours to fully charge a car at a rate of 1.4-3.3 kW \cite{Ustun2015ImpactSystems} (for reference, a typical laptop might draw 60-70 W while charging). Mode 3 requires dedicated supply equipment and is considered safer than both 1 \& 2 due to built-in safety systems such as auto-poweroff when not plugged in. Mode 4 is similar but requires even heavier equipment and uses its own equipment to convert AC to DC power. Both modes 3 and 4 ("fast charging") require anywhere from 50-250 kW \cite{Ustun2015ImpactSystems, Bossche2010ElectricInfrastructure} to charge, which needs specialized infrastructure beyond current residential or industrial outlets, but at a bonus of recharging a car in as little as 10 minutes. However, this might be overkill, since even most European consumers drive 50km/day or less and Israel drives even less than Europe, so "semi-fast" charging that draws up to 22 kW is probably suitable for most people \cite{Bossche2010ElectricInfrastructure}. Regardless, the Eilat electric grid is designed to accommodate typical residential loads such as the aforementioned 6,893 kWh/year, not thousands of 22 kW loads spread across the city, so the increased energy demand and strain on existing infrastructure would be enormous.

\subsection{Solar energy}
To help satisfy the enormous energy requirements of an EV powered city and reduce carbon emissions, a number of renewable energy options are available to Eilat and Israel as a whole, most notably solar power. Solar power requires large amounts of exposure to the sun and huge swathes of land to generate power effectively. Modern photovoltaic solar panels are only 10-17\% efficient \cite{Zhu2015DistributedGrid}), generating 150-200 watts per square meter, so supplying Israel with its average demand of 6.7 gigawatts would require 33-45 km$^2$ of contiguous solar panels, or widespread rooftop deployment to help satisfy a fraction of current demand \cite{Vardimon2011AssessmentIsrael}.

Unfortunately, the volatile nature of solar energy (affected by weather but entirely dependent on time of day, solar panels cannot generate power during the night) can cause problems with satisfying power demand and stabilizing power supply \cite{Lu2015IntroductionPEVs}, and the deployment of dedicated solar farms would likely require significant upgrades to the existing power grid \cite{Vardimon2011AssessmentIsrael} in both capacity and technology. Combined with the high energy demands imposed by a fleet of electric vehicles, a desperately-needed increase in solar power and 90\% EVs (as per Eilat's goal) can be expected to require significant changes to Eilat's electrical grid.

\subsection{Vehicle-to-grid charging}
One possibility for alleviating these concerns is the idea of vehicle-to-grid (V2G) charging. V2G charging centers around the idea of allowing two-way power flow between the grid and EVs connected to it for charging, instead of the traditional idea of one-way charging where you plug your car in and get a charge. Control and monitoring equipment built into the charger in collaboration with large scale grid monitoring could determine when power more readily available in the grid vs. when it's badly needed and decide whether the car charges its battery or discharges to help supplement the power supply. For instance, a well-known phenomenon is "peak time" when the work day ends, people go home, and power usage spikes. During these times, V2G-charging cars could serve as a distributed battery, discharging into the grid to relieve stress on power generators, while charging up during the nighttime in time to be driven in the morning \cite{Mahmud2015PowerEV}. While one car on its own might not have too much power, in large amounts the effect could be enormous. For reference, in 2017 Tesla built a 129 MWh battery in southern Australia which reduced service costs by 90\% \cite{Lambert2018Teslas90} --- if each EV had a battery capacity of 60 kWh, it would take exactly 2,150 of them to equal Australia's battery, or one EV for less than 5\% of Eilat's population. This makes V2G technology a serious option to mitigate a potential Eilat power crisis.

\section{Existing automotive industry}
Israel is not traditionally considered a strong player within the global automotive industry; however, they are proving to be a country on the rise in this regard. As a nation, Israel is known for the environment it has provided for technology start-ups to thrive. That in mind, many global brands are deciding to invest in the "high-end automotive technology sector." Some of these global brands include Intel, looking to apply its software automation in Jerusalem; Volkswagen AG, trying to popularize the ride hailing service called "Gett,"; even BMW AG, who made a large investment in a transit app called "Moovit" \cite{Coutinho2018IsraelIndustry}. 

With regard to the traditional facets of the automotive industry, the Israeli market also has companies specializing in Tier 1, 2, and Tier 3 supply; Tier 1 supplying companies with components for final assembly as well as aftermarket spare parts, Tier 2 supplying Tier 1 with sub-assembly components, and Tier 3 supplying simple engineered materials for Tier 2 suppliers, respectively. As of 2015, Israel has over 60 companies working in Tier 1 aftermarket supply creating spare parts and accessories such as batteries, ventilation, gears and valves. Israel also has over 50 companies that supply sub-assembly components to original equipment manufacturers (OEMs) \cite{MinistryofEconomyandIndustryStateofIsraelTheIsrael}. These companies may begin to shift over time as Israel seeks alternatives to traditional, inefficient cars, and moves toward lighter, smarter, and technologically savvy electric vehicles.

Keeping in mind that Israel is looking to move toward smart cars and efficient vehicles, there are a number of notable trends occurring in the world right now. One shift being made is the reduction of vehicle weight by using alternative materials, given that a car that is 10\% lighter would use 6-8\% less fuel \cite{MinistryofEconomyandIndustryStateofIsraelTheIsrael}. Other popular trends lie in the information technology sector, enabling vehicles to communicate with other vehicles and become more automated. These technologies include driver assistance systems, automated and smart transportation, and information security. All of these technologies aim to reduce the inherent error that human drivers have, decreasing commute times, reducing the impact of potential crashes, and keeping personal information out of the hands of hackers \cite{MinistryofEconomyandIndustryStateofIsraelTheIsrael}. Directly related to this increase in technology within the automotive sector, there exist some ethical considerations that must be taken into account. While the idea of more efficient vehicles would not receive much opposition, technology which takes control from the driver is heavily debated. Questions regarding the culpability of the driver in certain situations come into play, as well as driver rights, manufacturer liability, and implications for future generations.

Responding to this fundamental shift in the way the world is viewing transportation, Israel has responded with a few ideas of their own. For example, the trending navigation app "Waze" was based in Israel before they were bought out by Google in 2013. Waze encourages communication between drivers via a mobile app (not while driving obviously) where users can report traffic jams, vehicles stopped on the shoulder, hazards, and even police officers. Another Israeli company, "Cellint," monitors all active cell phones and detects their exact location on the roads every 30 seconds, collecting data to use for traffic management, research, and time-to-destination analysis. Other Israeli companies such as "Argus" aim to address the issue of information security, ensuring that data can be shared between vehicles seamlessly without interference from hackers \cite{MinistryofEconomyandIndustryStateofIsraelTheIsrael}. The list of Israel-based companies and their various applications goes on, however their goals all relate to creating safer, smoother and more efficient journeys for drivers.

Overall, the Israeli automotive industry is a developing one. This is not to say that it is underdeveloped, but that it has focused its lens on emerging technologies and brand new global needs. International companies have focused their attention to Israel in search of new viable automobile-related technologies. In addition to the cutting-edge technologies that have sprouted in Israel, they also have made efforts to vertically integrate many of them - ensuring that there is support from initial design to final production \cite{MinistryofEconomyandIndustryStateofIsraelTheIsrael}. Israel's location enables them to get the best of both commercial worlds; they can easily trade with Asia and Europe.

\section{Summary}

Overall, Israel stands to gain much by decreasing its oil reliance, increasing solar power production, and switching to electric vehicles. In the process it would gain a more reliable power system, greener vehicles, and reduced political difficulties involved with importing oil from its less-than-friendly neighbors. However, the process of making the switch over the next 20 years can be expected to be complicated, with numerous technological developments required, in-depth study of the social effects and social feasibility of electric vehicles, and possibly mitigation to avoid threatening the Israeli automotive industry. Additional research and investigation is required to determine the feasibility and implications of Eilat's plan to switch to 90\% electric vehicles by 2040.\

\newpage
\chapter{Methods}
\textit{Not part of this submission.}

% Bibliography -- automatically managed in APA style, on a new page
\newpage
\bigskip
\bibliographystyle{apacite}
\bibliography{refs}

\end{document}

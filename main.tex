\documentclass[12pt]{article}                         % Make it an article
\usepackage{indentfirst}                       % All paragraphs indented
\usepackage{setspace}                           % For 1.5 spacing
\usepackage[hidelinks]{hyperref}      % Make table of contents clickable
\usepackage[margin=1in]{geometry} % 1 inch margin
\usepackage[english]{babel}                     % For "fancy" quotes
\usepackage[autostyle, english=american]{csquotes} % Same
\usepackage{apacite}                            % Use APA citations
\usepackage{etoolbox}                           % Misc
\usepackage{titlesec}
\usepackage{listofauthorships}
\MakeOuterQuote{"}                              % For fancy quotes
\titlespacing*{\section}{0pt}{0pt}{0pt}
\titlespacing*{\subsection}{0pt}{0pt}{0pt}
\titlespacing*{\subsubsection}{0pt}{5pt}{-3pt}

\title{Converting Eilat to Electric Vehicles}
\author{Evan Goldstein, Valeria Kopper, Christopher Myers, Zachary Zlotnick}
\date{November 2018}

\begin{document}
\pagenumbering{gobble}
\maketitle
\newpage

\renewcommand\abstractname{Summary} % "Abstract" -> "Summary", the header lies

\pagenumbering{roman}
\tableofcontents
\newpage
\listofauthorships
\newpage
\pagenumbering{arabic}
\doublespacing

% Actual content begins here!
\section{Introduction}[Christopher M., Zachary Z., Evan G., Valeria K.]

Modern Israel relies on combustion powered vehicles for transportation. With nearly 70\% of Israeli citizens driving to work and another 18\% taking buses, commute times have increased 30\% over the past decade \cite{Dori2018IsraeliRoads}, increasing Israel's dependence on oil. The environmental impacts of oil usage are well known, but Israel produces only a few thousand barrels of oil per day. Therefore it must import 99\% of its oil \cite{Engber2006WhereOil}, mostly from Egypt, Russia, and nearby Arabic nations. Considering Israel's complicated foreign relations and oil crises \cite{Engber2006WhereOil}, this presents serious motivation to decrease oil reliance in favor of local power sources, such as solar power.

To that end, Eilat is investigating the promotion of electric vehicles (EVs) along with related infrastructure and charging systems. Electric vehicles are vehicles that use electric motors instead of engines, drawing power from a battery instead of fuel. Electric cars are dramatically more energy efficient, with a range of up to several hundred kilometers \cite{Friel2010ManagementVehicles}, even in lower end models.

The benefits and drawbacks of EVs from an economic, environmental, and engineering perspectives have been thoroughly researched. The energy efficiency of electric cars is known but the overall energy cost varies with the source of the energy. That is, if the energy is sourced from renewable sources (e.g. solar, the current best option for Eilat), all-electric is advantageous. On the other hand, if the energy is sourced from fossil fuels like oil, it is more efficient to generate power on-board, as in a hybrid-electric vehicle \cite{Dincer2010EconomicOptions}; this also applies to environmental effects. Eilat is also investigating vehicle-to-grid (V2G) charging, or the idea of allowing two-way power flow between a car and the electric grid. With V2G, a car battery can supply power back to the grid when needed. Unfortunately, the complicated technology required is still under development \cite{2018Nuvve...}.

Applying efforts under development elsewhere to Eilat's situation does not take into account Eilat's demographics and power sources. Eilat is a relatively small city with a population of about 50,000, with a well-connected city center (you can walk almost anywhere) and only one bus company \cite{TransportationEilat}. While 70\% of commuters drive to work, Eilat sees up to 3 million tourists per year, which will likely affect Eilat's plan. Furthermore, if Eilat aims to transition to electric vehicles \textit{and} increase solar usage, this can interfere with the benefits of V2G. Under normal circumstances charging would take place at times of decreased load, usually at night when solar panels cannot generate power. Even without taking V2G into consideration, serious upgrades to Eilat's power grid will be required just to increase solar power \cite{Vardimon2011AssessmentIsrael}. On average consumers are unwilling to pay more in the name of environmental friendliness \cite{Mock2010MarketVehicles}; without an understanding of what could motivate Eilat citizens to switch to electric cars, it may prove difficult to determine proper incentives. Finally, Eilat hasn't been subject to electric car development before either, so there are no charging stations and no information on where they could go.

%start here
The goal of this project is to assess the technological viability and requirements of Eilat's goal to have 90\% electric vehicles by 2040, and develop a roadmap for its implementation in order to aid the city of Eilat in switching to electric vehicles. To accomplish this we aim to determine what the most effective economic incentive to switch is, to determine the maximum price that Eilat consumers would be willing to pay, what the maximum range of the cars should be, and investigate the potential placement for charging stations in Eilat. Although adapting the technology to Eilat's situation and convincing citizens to convert to EVs may prove challenging, it opens unique opportunities. Eilat's effort can prove V2G's potential, demonstrate a working EV-powered city, and act as a role model for other cities looking to switch to a majority of EVs.

\newpage
\section{Background}
To develop a roadmap for Eilat's plan, it is necessary to have a better understanding of transportation in Eilat, the technological details behind electric cars, past efforts, the economics of Israel's dependence on oil, and possible incentives that Eilat implement to motivate its citizens to switch to electric cars.

\subsection{Transportation}[Evan G.]
Eilat is a relatively small city with a population of 47,800. Within the city center, you can walk almost everywhere. There is only one bus company, Egged. \cite{TransportationEilat} "Get Taxi", an Uber-like service, is also available. 

\subsubsection{Cars and Public Transportation}
In March 2018, a report published by the Bank of Israel found that commute times in the country have increased by 30\% in the last decade. This is likely because private and commercial vehicles account for how 69\% of Israelis get to work nationwide. Following that, 18\% take a bus. Tel Aviv and Haifa are very close to the national numbers, but in Jerusalem, the statistic changes drastically, to 54\% car and 32\% bus \cite{Dori2018IsraeliRoads}. By comparison, in European cities like Barcelona and Brussels public transportation accounted for 40\% of all commuting and in Paris, 70\%. The report found that those who do take the bus to work do so not because it is convenient or efficient, but because they cannot afford to own a car, and therefore have no choice. The report came after the International Monetary Fund said Israel’s transportation infrastructure was insufficient and was threatening to be a drag on economic growth. 

There is little data on public transportation specific to Eilat. It's important to note that a high volume of tourism makes Eilat is a very different city from the likes of Tel Aviv and Haifa. A high percentage of those driving and using public transportation in and around the city are tourists, and any transportation development must take this into consideration.

\subsubsection{Energy waste due to traffic}
The stop and go movements of vehicles in heavy traffic is a major case for the use of electric vehicles. A gas-powered vehicle is inefficient while idling. In the US, this accounts for 10.6 billion gallons of fuel per year \cite{Carrico2009CostlyVehicles}, or 7.4\% of all gasoline consumption (given 143 billion gallons total annual gasoline consumption \cite{Administration2018HowConsume}). Electric vehicles would eliminate this wasted power by using no energy while stopped. Additionally, when slowing down, EVs could recharge their batteries by converting their kinetic energy to stored electrical energy through a process called "regenerative breaking."

\subsubsection{Car Ownership}
Israeli car ownership has increased rapidly in the past 20 years. In 1995, car ownership was climbing at 6-7\% per year --- among the fastest in the world at the time \cite{Slater1995IsraelCulture}. A 2012 report from the Israel Tax Authority showed that car owners in the center and north of the country drove the least --- around 14,000 kilometers a year on average between the Tel Aviv, Jerusalem and Haifa areas --- while car owners in far-north Metula area drove the most, at 22,000 kilometers a year. Eilat car owners, who are the farthest from the country’s center, drove the second most at 19,000 kilometers a year \cite{Schmil2012WhatHow}. The report states that between 1995 and 2011, tax revenue from cars more than doubled. Additionally, whether due to increasing green-awareness or Israel’s Green Tax Reform, cars on Israel's roads are becoming less damaging to the environment\cite{Schmil2012WhatHow}.

\subsection{Electric Vehicles and Solar Power}[Christopher M.]
\subsubsection{Electric Vehicles}
Electric vehicles (EVs) are an energy efficient alternative to traditional internal combustion engine vehicles (ICEVs), relying on built in lithium-ion batteries to store power and electric motors to generate traction to move the vehicle. ICEVs produce carbon dioxide (CO$^2$, a known greenhouse gas) emissions from burning gasoline, but convert 10-15\% of the chemical energy within into traction that moves the car. In comparison, a hybrid vehicle --- combining both combustion and EV technology --- can yield 30-40\% efficiency \cite{Zhu2015DistributedGrid}. For a more concrete example, gasoline has a known energy density of 9.5 kWh/liter \cite{EngineeringFactors}, and a 2018 Honda civic sedan has 46.9 liter tank with a fuel efficiency of 14.9 km/l \cite{20182018Information}. This means it can store 445.5 kWh with a range of 698 km. In comparison, a Tesla Roadster has a max range of 393 km but a battery capacity of only 60 kWh \cite{Friel2010ManagementVehicles}, making the Roadster over 4 times more energy efficient despite its lower range.

As can be seen, EVs are currently limited in both battery capacity and range, presenting a usage-awareness challenge to consumers who wish to switch. On the low end of the scale, cars can have as little as 18.8 kWh batteries with a range of only 160 km, but even midrange cars have 70 kWh batteries and a range of 500 km \cite{Glon2018HeresPerformance}. To get an idea of the scale of the power here, Israelis use about 7,000 kWh in electric power per capita per year \cite{2018Key2018}, so a single electric car charge can be up to 1\% of a year's power --- trivial for a single charge but over time it easily adds up. The key factor in EV range is battery capacity, which unfortunately remains expensive at up to 1,000 USD/kWh in 2010 \cite{Mock2010MarketVehicles} although future projections estimate that \$200-300/kWh is a realistic goal and mass production will further reduce expense. Range is a known concern of drivers, who on their own have little incentive to switch to EVs \cite{Mock2010MarketVehicles}, so the best possible vehicle range and availability of charging stations is essential.

\subsubsection{Charging and Power Requirements}
EVs have significant power requirements while charging that require special accommodations. Chargers are broken down into levels 1-4, with increasing charge rate and infrastructure requirements as you progress along the scale. Modes 1 and 2 are most common currently and consist of simply plugging the EV into AC power (mode 2 has a special in-line control box for added safety) \cite{Bossche2010ElectricInfrastructure}, using hardware built into the car to convert AC to DC power. These chargers are considered less safe but are ideal for opportunity charging (plugging in your car whenever parked for an extra top-up) and don't require special infrastructure, although they can take up to 18 hours to fully charge a car at a rate of 1.4-3.3 kW \cite{Ustun2015ImpactSystems} (for reference, a typical laptop might draw 60-70 W while charging). Mode 3 requires dedicated supply equipment and is considered safer than both 1 and 2 due to built-in safety systems such as auto-poweroff when not plugged in. Mode 4 is similar but requires even heavier equipment and uses its own equipment to convert AC to DC power. Both modes 3 and 4 ("fast charging") require anywhere from 50-250 kW \cite{Ustun2015ImpactSystems, Bossche2010ElectricInfrastructure} to charge, which needs specialized infrastructure beyond current residential or industrial outlets, but at a bonus of recharging a car in as little as 10 minutes. However, this might be overkill, since even most European consumers drive 50km/day or less and Israel drives even less than Europe, so "semi-fast" charging that draws up to 22 kW is probably suitable for most people \cite{Bossche2010ElectricInfrastructure}. Regardless, the Eilat electric grid is designed to accommodate typical residential loads such as the aforementioned 6,893 kWh/year, not thousands of 22 kW loads spread across the city, so the increased energy demand and strain on existing infrastructure would be enormous.

\subsubsection{Solar Energy}
To help satisfy the enormous energy requirements of an EV powered city and reduce carbon emissions, a number of renewable energy options are available to Eilat and Israel as a whole, most notably solar power (considering Eilat's sun exposure and low precipitation). Solar power requires large amounts of exposure to the sun and huge swathes of land to generate power effectively. Modern photovoltaic solar panels are only 10-17\% efficient \cite{Zhu2015DistributedGrid}), generating 150-200 watts per square meter, so supplying Israel with its average demand of 6.7 gigawatts would require 33-45 km$^2$ of contiguous solar panels, or widespread rooftop deployment to help satisfy a fraction of current demand \cite{Vardimon2011AssessmentIsrael}. Considering that Israel is already a small nation and electric cars will massively increase electric consumption, finding space for new solar farms can be challenge.

Unfortunately, the volatile nature of solar energy (affected by weather but entirely dependent on time of day, solar panels cannot generate power during the night) can cause problems with satisfying power demand and stabilizing power supply \cite{Lu2015IntroductionPEVs}, and the deployment of dedicated solar farms would likely require significant upgrades to the existing power grid \cite{Vardimon2011AssessmentIsrael} in both capacity and management technology. Combined with the high energy demands imposed by a fleet of electric vehicles, a desperately-needed increase in solar power and 90\% EVs (as per Eilat's goal) can be expected to require significant changes to Eilat's electric grid.

\subsubsection{Vehicle-to-Grid Charging}
One possibility for alleviating these concerns is the idea of vehicle-to-grid (V2G) charging. V2G charging centers around the idea of allowing two-way power flow between the grid and EVs connected to it for charging, instead of the traditional idea of one-way charging where you plug your car in and it charges. Control and monitoring equipment built into the charger in collaboration with large scale grid monitoring can determine when power is more readily available in the grid vs. when it's needed elsewhere and decide whether the car charges its battery or discharges to help supplement the power supply. For instance, a well-known phenomenon is "peak time" when the work day ends, people go home, and power usage spikes. During these times, V2G-charging cars could serve as a distributed battery, discharging into the grid to relieve stress on power generators, while charging up during the night in time to be driven in the morning \cite{Mahmud2015PowerEV}. While one car on its own has limited power, in large amounts the effect could be enormous. For example, in 2017 Tesla built a 129 MWh battery in southern Australia which reduced service costs by 90\% \cite{Lambert2018Teslas90} --- if each mid-range EV had a battery capacity of 70 kWh, it would take exactly 1,843 of them to equal Australia's battery, or one EV for less than 4\% of Eilat's population. This makes V2G technology a serious option to mitigate an Eilat power crisis.

\subsection{Existing Automotive Industries and Prior Efforts}[Zachary Z.]

\subsubsection{The Existing Automotive Industry}
Israel is not traditionally considered a strong player within the global automotive industry, as they have continued to import most of their automobiles from other countries. As of 2015, 254,000 vehicles were imported to Israel - only to increase by 12\% in 2016 to a total of 286,728 vehicles \cite{Halavy20172016Imports}. In addition, 1,004,019 of the vehicles on Israel's roads as of 2017 were 3 years old or under \cite{Halavy20172016Imports}. This constant influx of cars due to an assumed increase in Israeli wealth has caused the few manufacturers that reside in Israel to become obsolete compared to other big manufacturers abroad including Hyundai, Toyota, and Kia \cite{CarLoanWorld2015TheAustrailia}. Numbers like these would come as a shock to most when they consider that Israel imposes a high tax rate, as discussed later, on any imported vehicle. This is relevant to the future of EVs in Eilat, because they will most likely be manufactured abroad in European nations, as most of Eilat's current vehicle composition is. 

%All of this being said, Israel does have a reputation as a breeding ground for start-ups to thrive. One example of a forward-looking start-up is "Mobileye," who has partnered with Intel and BMW to create a fully autonomous car by the year 2021. This would seemingly reduce traffic buildup in busy areas, reduce crash rates, and ensure a quicker commute \cite{Lierbermann2016ThreeCar}. Ethical considerations would need to be taken into account however, and it would most likely only encounter widespread adoption once the cars are exponentially better drivers than humans. While the technology is not there yet, the Israeli firm has already begun testing these and it should only be a matter of time before they are available to the public.  

\subsubsection{A ``Better Place''}
With regard to EVs, there has been little effort made within Israel following the efforts made by Shai Agassi back around 2007. Agassi was a entrepreneur who had seen a future of electric vehicles in Israel - creating a start-up called "Better Place" which aimed to "eliminate range anxiety and to deliver cars cheaply" \cite{ChafkinAGoing}. Though these goals were attainable, the reason they failed can be attributed to the development of lofty goals along the way, and the over-confidence of CEO Shai Agassi. He was often documented to have used "Shai math," the process of taking an entirely realistic metric and making it nearly impossible to attain. Other flaws in the "Better Place" business plan lied in their allocation of investments and resources. For example, they decided to spend \$60 million on a navigation system for their vehicles, which could have been licensed to TomTom for \$29.95 per vehicle. Another instance of this was when they installed charging stations. Shai projected the charging stations to cost approximately \$500,000 when all was said and done, when in reality it cost \$2,000,000 per charging station - a 300\% difference \cite{ChafkinAGoing} This is not to say that the misinformed and erratic business decisions, however it proves to be relevant when looking at the technological side of "Better Place." 

When looking at the actual implementation that Shai Agassi and the venture capitalists, it is important to note that their technological vision was sound. They wanted to create a cheap family car that had enough range for most people, and to place charging stations in sites intelligently. They imported their vehicles from a foreign manufacturer, Renault, in order to reduce the required overhead. The lithium battery would be replaced in about 3 minutes, about the same time as a gasoline fill up. Their vision was entirely realistic, and had they stuck to their initial business plan, "Better Place" could have become a mainstay in Israel - now 5 years removed from it's failure. With all of this in mind, their initial model and technological ideas will provide achievable goals and ideas for the future of EVs in Eilat.

\subsection{Israeli Economics}[Valeria K.]
\subsubsection{Oil Dependence}
Currently the State of Israel has a strong dependence on oil imports, which has contributed to already complicated ties with its neighboring nations in the Middle East. Israel produces incredibly small quantities of oil, as little as only a couple thousand barrels of oil a day \cite{Engber2006WhereOil}. Such a small oil production does not allow Israel to sustain more than 1\% of its oil needs, which means it imports over 99\% of its oil \cite{Engber2006WhereOil}. This generates an undesired and convoluted vulnerability for Israel. Since its establishment, the State of Israel has continuously sought oil resources closer to home in order to increase stability and safety as well as decreasing import costs. Its overall growth and increase of oil demand has further fostered Israel's need to decrease its dependence on other nations for almost the entirety of their oil resources. 

Ideally, Israel can decrease its dependence on other nations by decreasing its oil consumption. Focusing on the development of energy consumption in future years progressing towards the use of renewable energy instead of oil consumption would yield great benefits in terms of environmental, social and political aspects. This can be achieved by focusing on renewable energy production and shifting towards electric machines, such as EVs, versus oil-run machines. Israel has the opportunity to expand upon its untapped potential for renewable energy production, especially solar polar, which would grant Israel an increased independence from its oil imports and undesired ties to neighboring countries.

\subsubsection{Economic Incentives}
An important aspect to take into account in regards to implementation of EVs and V2G in Eilat is the use of economic incentives. An economic incentive can be understood to be a something that influences people in a way that causes them to behave in a certain way or make certain decisions. This is often adopted by governments in order to encourage behavior from their countries populations, and the implementation of EVs is the perfect example for this. The city’s population’s willingness to adopt EVs is what ultimately determines whether it can have a 90\% EV usage by 2040. One of the most common economic incentive is tax cuts, in this case decreasing purchasing taxes or allowing for full purchasing tax exemptions (for EVs and not for traditional cars) can be incredibly effective \cite{Katsovitch2011JIMS:Highest}. Taxes on cars in Israel are among the highest in the world at 83\%, which is five times higher than most European countries. It is estimated that it can take up to 28\% of the average exempt from road tolls. Additionally, the government could potentially allow charging stations installed around the city to be free of charge, possibly for a predetermined time period. These charging stations could also be initially free and then the consumer could increasingly (likely percentage wise) start paying for public electric charge over time. The government can also potentially offer reduced electric billing for the homes of owners of EVs. Electric charging would represent significant savings for citizens by eliminating their personal oil costs, which also serves as an important financial incentive.

Additional to economic incentives the government can also implement non-monetary incentives. A group of EV owners in Norway said the strongest incentive for them was bus lane access \cite{Bjerkan2016IncentivesNorway}. In the state of California EV users are allowed free municipal parking and access to the carpool lane \cite{TeslaElectricIncentives}. Ireland plans on implementing priority parking for EVs \cite{SustainableEnergyAuthorityofIreland2011ElectricRoadmap}.

Among these various forms of economic and non-monetary incentives there can be clear distinctions among different groups in the population. Factors such as gender, age, and education influence people in regards to what would incentivize them to purchase an EV. Identifying these and the composition of a population is crucial to determining what the most effective use of incentives for a particular city or country is. It is important to note that this does not address consumer attitudes and behavioral changes, these are implied as repercussions that will follow policy implementation as shown in other countries such as Ireland and Norway\cite{SustainableEnergyAuthorityofIreland2011ElectricRoadmap,Bjerkan2016IncentivesNorway}. Overall if the Israeli government intends to achieve having EVs as the city's major form of transportation it must be backed up by policy (most likely monetary policy) and supportive measures (such as non-monetary incentives) for the population.

\subsection{Summary}[Christopher Myers]
Overall Eilat stands much to gain by switching to electric vehicles, including decreasing its oil reliance, encouraging a reliance on Israeli-produced energy, reduced emissions, and increased political independence. However, the process of switching to 90\% electric vehicles by 2040 can be expected to be complicated. Since Eilat obviously cannot directly control its citizens purchases, the city must find a way to encourage drivers to switch to electric vehicles, using an in-depth understanding of the desired prices and vehicles ranges. In the event that the switch \textit{does} happen, Eilat can expect difficulties related to charging thousands of electric vehicles, namely problems with its power supply and electric grid. Charging cars incur a heavy electric cost, so not only must Eilat increase its solar power generation, but it must upgrade its electric grid and investigate new locations for charging stations. More research is required to develop a roadmap for Eilat's plan.

\newpage
\section{Methods}
The goal of this project is to assess the technological viability and requirements of Eilat's goal to have 90\% electric vehicles by 2040, and develop a roadmap for its implementation in order to aid the city of Eilat in switching to electric vehicles. To accomplish this we aim to determine what the most effective economic incentive to switch is, to determine the maximum price that Eilat consumers would be willing to pay, what the maximum range of the cars should be, and investigate the potential placement for charging stations in Eilat.

% Write intro here

\subsection{Determining Economic Incentives}[Valeria K.]
\subsubsection{Surveys and Data Analysis}
Determine the most effective policy to incentivize the population towards transitioning to 90\% EV usage the demographics of the population of Eilat need to be broken down. To do so we plan to analyze data provided by the sponsor. By doing so we will understand the make up of the population of Eilat. Additionally, we will compare this to data regarding which population segments are the major car consumers. The overlap between these two will yield the population we want to target. Once we have this we can survey members that fall into this category in order to understand what would motivate them to purchase an EV instead of a conventional car.

One of the most challenging and crucial aspects of developing a roadmap in order to achieve having 90\% EVs in Eilat is understanding the intricacies of the human aspect of the implementation of EVs. The technology behind it is relatively straight forward; however, people’s reactions can be much more convoluted. In order for a roadmap like this one to succeed it must address this and evaluate what is the best way to get people to embrace EVs. Focusing on this aspect through an economic scope can allow for effectively targeting the population and determining the most effective incentives available. 

Population groups determined by factors such as age, gender and education have varying reactions to incentives to adopt EVs. Determining the makeup of the population in Eilat is thus crucial to understanding what incentivizes each population segment. Utilizing various incentives targeting the city’s specific demographics will likely yield the best results. In order to do so our group will have to determine the population make up through the analysis of data provided by the project’s sponsor. This will help us understand which groups make up the majority of the population in Eilat and which groups make up the majority of car purchases. Comparing the majority of the composition of the population in Eilat to the majority of car consumers will be key to determining what the target population groups are.

Once we determine which population groups are the most relevant we plan on surveying people based on this. The purpose of the surveys is to establish what the most effective incentive is for each of these sub-populations. Given that the specific demographic determinations have not been made, the surveys will change and adapt while we are in Israel. We will focus the surveys on understanding  people's preferences. Through understanding the demographic composition of Eilat and understanding what appeals to the major car consumers we can determine which is the most effective economic incentive or combination of incentives. 

\subsubsection{Deliverables}
Through the preformed analysis and the data collected from the surveys we will be able to construct a plan of which incentives, both financial and non-monetary, are the ideal combination for Eilat to promote the incorporation of EVs. This will consist of formulating a proposal of a policy the government can implement. This can be part of a timeline that will be a part of the overall roadmap we are creating. 

\subsection{Determining Ideal Electric Car Price}[Zachary Z.]

\subsubsection{Surveys \& Data Analysis}
In order to determine the best price for electric vehicles, we need to gather data from potential consumers regarding their willingness to purchase a vehicle at a given price. We will give the survey 5-10 different prices and a range of 350km, and for each price they will indicate "yes" - I would purchase an EV for that price, or "no" - I would not.  In order to determine prices that are worthwhile to test we will look at prior implementations of EVs and create an appropriate range that spans the lowest and highest prices we find. We do not know yet the demographic of the participants since we are awaiting data from sponsors, and have not decided concrete prices to try. These prices should be given with as little involvement as possible with relation to incentives for purchasing EVs, so that we are not yet sure what would be most effective, and what the government would be willing to employ.

Most of the work with regard to finding a target price falls in analyzing the data properly. This would begin with determining a percent purchase rate for each given price. Consider a basic example where we survey participants on prices ranging from \$30,000-35,000 (incrementing by \$1,000 yields 6 prices). If we were to take the total number of people who say "yes" for any of the 6 prices, and divide it by the total number of participants, then we can figure out the percentage of people for each price range. This would yield a table of percentages of consumers who would pay each price.

In order to determine target price it should be relatively straightforward. We take the price which is closest to 90\% adoption without exceeding it, and use that as a baseline price for an everyday person's EV. This being said, the percentage can be significantly lower if we were to add economic incentives, both monetary and non-monetary. This is where we can use the data we have obtained from the economic incentives surveys to determine which incentives will increase adoption to 90\%. 

\subsubsection{Deliverables}
With comprehensive information on Eilat citizens' willingness to pay for electric vehicles, we will be able to construct a table which depicts the predicted adoption of EVs. Though EVs can vary in range and size, it will provide a baseline that is both based on participant opinions, as well as other implementations across the globe. From this price, we can determine how economically feasible it is for manufacturers to produce EVs, and the kinds of incentives we will need to offer for them to become popularized across the Eilat transportation sector.

\subsection{Determining Acceptable Electric Car Ranges}[Christopher M.]
\subsubsection{Surveys \& Data Analysis}
To gain an understanding of the range of electric cars that consumers expect and would be best served by, we will conduct surveys of the Eilat population. These surveys will be random samples of the Eilat population as a whole without regard to economic status or occupation as we are most interested in the opinions of the general population of Israel. However, it is well known that the range that drivers would \textit{prefer} is different than the range that drivers could reasonably live under ("range anxiety"), so the surveys will collect data on both preferred ranges and minimum acceptable ranges. This will be surveyed alongside questions of electric car price in order to determine whether consumers would be willing to accept a lower range car in exchange for reduced cost.

More specifically, the surveys will ask questions about Israeli driving habits (e.g. "How many kilometers do you drive per week on average?") first, move on to asking about long trips and frequency thereof, then suggest a list of possible ranges (0-75 km, 75-150 km, etc.) and ask what they need at a minimum and what they would prefer. The first two subjects are not intended for genuine data gathering but instead will encourage the participant to reflect on their actual usage of their vehicle. Combined with a brief overview of electric car costs and their ranges (low end to high end models), this should help participants right range anxiety and give more conservative estimates. Care should be taken to ensure that price ranges shown are not too intimidating to the average consumer; that is to say, there is no need to show the price of a high end luxury electric car if only a fraction of the population could afford it.

Data gathered from these surveys will most importantly consist of information on the minimum acceptable ranges for Eilat citizens and what their preferred range would be, with the secondary bonus of up-to-date information on Eilat citizens' driving habits. Averaging these values and calculating basic statistics information (median, standard deviation, etc.). The gap between minimum and preferred range is also of use, as it serves as a good indicator of Eilat citizen's vulnerability to range anxiety. For instance, if minimum range is on average a third of what preferred range is, we could anticipate heavy resistance to roll-out of early (modern) EVs on the basis that most citizens would find their ranges too limiting. If, on the other hand, we find that the two are very close, it indicates little to no range anxiety and the question of EV roll-out becomes less a question of technology delays and more a question of consumer economics. Considering that batteries and other EV technology remains under constant development, this could prove a serious boost to Eilat's goals.

\subsubsection{Deliverables}
With up-to-date information on Eilat citizens' travel habits and more importantly their minimum \& preferred ranges, we can construct a more effective timeline as part of our roadmap for Eilat's 2040 goal. This information can be combined with Eilat demographics and known EV ranges and prices to determine when it's most likely that Eilat citizens could start switching to EVs. For instance, if the average Eilat citizen wants to be able to drive at least 150 km in one trip but affordable EVs that can manage that aren't expected until 2030, we could set a definite point on our timeline. This point in time may prove to be a turning point in Eilat's plan, so depending on the results of our analysis, we may recommend preparation leading up to this point and action beginning immediately at it.

\subsection{Analyzing Infrastructure and Planning Upgrades}[Evan G.]
Electric vehicles will require gas stations to be upgraded to EV charging stations. Transitioning from ICEVs to EVs will require an increase in electricity production to cope with the power requirements of EVs. The most effective way to increase electricity production will be to build additional solar farms. Eilat's electrical grid will require upgrades to handle the additional load of EV charging stations. 

\subsubsection{Charging Station Locations}
We will determine the areas which will need the highest density of charging stations (charging stations per unit of area) by analyzing traffic data and locations of existing gas stations. High-traffic areas will need a higher density of chargers. We will determine which locations need fast chargers and which locations people will be parked long enough to use a slower charger which draws less power from the grid. Not all EVs will have the range to drive from Eilat to Tel Aviv or Jerusalem on a single charge, so we will also determine the optimal locations for placing chargers along route 40 considering population density and existing infrastructure along the route.

\subsubsection{Solar Energy Expansion}
 We will use an algorithmic approach to determine the optimal size and location for new solar farms. We will consider factors of available of land such as cost, distance from the city, and power loss over transmission lines. We will estimate the power requirements of the planned EV chargers, given how many we deploy and how much power they will require.
%multi-objective Bees Algorithm 

\subsubsection{Electric Grid Upgrades}
Once we have determined the power requirements of the planned EV chargers and how much new solar will be required to meet the demand, we will determine what upgrades to the city's electrical grid will be required to handle the increased load. We will investigate the current state of the grid, determine the weakest points and most critical upgrades. We will consider desired capacity and ease of future upgrades. 

\subsubsection{Deliverables}
We will create a plan for the installation and distribution of EV charging stations. We will also plan the upgrades and additions to the electrical infrastructure in and around Eilat required to support the load of the charging stations.

\subsection{Conclusion}[Christopher M.]
By conducting comprehensive surveys of Eilat's citizens to determine their EV requirements in range and price, gathering feedback on EV incentive options, and investigating locations for new charging stations and solar power plants, we should be able to combine information to create a comprehensive roadmap. Researched information on EV development, range, and prices over time combined with the EV range + price requirements we can determine if and when the population will begin switching and at what rate adoption will occur. Using that we can establish a timeline for how Eilat should begin deploying charging stations, when they should activate incentives, and when pressure on the power grid and power generators will require immediate action. Finally, based on our predicted adoption rates and research into V2G charging and EV efficiency we can estimate overall energy savings and efficiency increase that Eilat can expect to see.

% Bibliography -- automatically managed in APA style, on a new page
\newpage
\bibliographystyle{apacite}
\bibliography{refs}

\end{document}

\documentclass{article}                         % Make it an article
\usepackage{indentfirst}                       % All paragraphs indented
\usepackage{setspace}                           % For 1.5 spacing
\usepackage[hidelinks]{hyperref}      % Make table of contents clickable
\usepackage[margin=1in]{geometry} % 1 inch margin
\usepackage[english]{babel}                     % For "fancy" quotes
\usepackage[autostyle, english=american]{csquotes} % Same
\usepackage{apacite}                            % Use APA citations
\usepackage{etoolbox}                           % Misc
\MakeOuterQuote{"}                              % For fancy quotes
% Turn off URLs in citations. 
% https://tex.stackexchange.com/questions/425392/how-remove-links-from-apacite
\renewenvironment{APACrefURL}[1][]{}{}
\AtBeginEnvironment{APACrefURL}{\renewcommand{\url}[1]{}}

\title{Israel IQP Project Proposal}
\author{Evan Goldstein, Valeria Kopper, Christopher Myers, Zachary Zlotnick}
\date{November 2018}

\begin{document}
\pagenumbering{roman}
\maketitle

\renewcommand\abstractname{Summary} % "Abstract" -> "Summary", the header lies
\begin{abstract}
This short section at the top should be an executive summary of the rest of the document, something like an abstract. It should be very short (one paragraph at most) and extremely concise, covering the proposal at a very high level. It may be appropriate to put sponsor acknowledgments here if they're needed. \textbf{\textit{This section might not be needed!}}
\end{abstract}

\tableofcontents
\newpage
\pagenumbering{arabic}

% Actual content begins here!

\section{Introduction}
This section should give an overview and explanation of the problem, including a concise problem statement. The last paragraph of the introduction should be the goal statement, which covers what we aim to accomplish, who we're assisting locally, and the problem itself. This section should be 1-2 pages long.

\newpage
\section{Background}
The background section should cover the current state of affairs in Israel and should give detailed background information on which we should base our methods. For instance, a discussion of electric car prevalence in Israel, or the capacity of Eilat's power grid, and so on.

\subsection{Electric cars \& vehicle-to-grid charging}
Details on the current state of electric cars in Israel (esp. Eilat), what vehicle-to-grid charging is, etc. It might be a good idea to give some background on the technical details of vehicle-to-grid charging in particular, as long as it's simple enough. Information about Israel's solar plans might be good to put here.

\subsection{Existing economy}
Information about the current makeup of Israel's automotive industry (if they have one) and peripheral industries such as car repair or manufacturing. A discussion of the usage of the costs of oil imports might also be relevant.

\subsection{Transportation}
This should be the social effects situation, but since we haven't conducted any research on what consumers in Israel think of electric cars, we can't call it that. Basically this should cover the travel habits of modern Israeli citizens, esp. in Eilat. For example, how often do they drive? What's the average length of a trip? How much does the average person spend on gas?

\subsection{Oil imports \& Israeli independence}

A discussion of the extremely complicated relations between Israel and its neighbors could be useful. We should discuss where Israel gets its oil from in particular, and what kind of effect that has on their relations with other countries.

\newpage
\section{Methods}
The methods section should explain what we actually want to \textit{do} in Israel to address our goal statement. Since we haven't really established that yet, these headers should serve as a guideline and not a definitive outline of what we're going to do.

\subsection{Economic research (?)}
What are we going to do to investigate the economic impact of electric cars and vehicle-to-grid charging? Can we expect to see job losses or job gains? Will Israeli citizens be able to afford to switch between now and 2040?

\subsection{Social effects research (?)}
What are we going to do to find the social effects of electric cars? Maybe we should survey people? Maybe we should look at traffic patterns? No idea.

\subsection{Political effects research (?)}
How are we going to figure out the political and foreign relations effects that going electric will have? Will we look at how reducing oil usage will increase Israel's independence? Will it make them more militarily secure?

\subsection{Environmental effects research (?)}
How will we determine the environmental effects of electric cars vs. traditional gas powered cars? Will we collect data ourselves? Will we use existing research? We could probably share data between this and the other sections to, for instance to figure out how many cars are on the road.

\subsection{Technological feasibility}
How will we determine the technological feasibility of Eilat's vehicle-to-grid charging station?

\bigskip
This section is here to demonstrate how citations and bibliographies work in Overleaf/\LaTeX; if we don't cite things, they naturally won't appear in the bibliography, so here's \textit{all} our citations...

\cite{Cheslow2011BetterCountry, Zhu2015DistributedGrid, Bennett2015EconomicGrid, Shinde2018ElectricInfrastructure, Ustun2015ImpactSystems, Ustun2015ImpactSystems, Lu2015IntroductionPEVs, OraCorenIsraelLane, Larkin2008IsraelsDependence, IsraeliMinistryofEnergyIsraelsEconomy, Leskarac2015PEVOn, Rahman2015PowerEVs, Rahman2015PowerEVs, Mahmud2015PowerEV, Mullan2012TheConcept}

% Bibliography -- all automatically managed in APA style, on a new page
\newpage
\bibliographystyle{apacite}
\bibliography{refs}

\end{document}

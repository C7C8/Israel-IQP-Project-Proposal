\documentclass{article}                         % Make it an article
\usepackage{indentfirst}                       % All paragraphs indented
\usepackage{setspace}                           % For 1.5 spacing
\usepackage[hidelinks]{hyperref}      % Make table of contents clickable
\usepackage[margin=1in]{geometry} % 1 inch margin
\usepackage[english]{babel}                     % For "fancy" quotes
\usepackage[autostyle, english=american]{csquotes} % Same
\usepackage{apacite}                            % Use APA citations
\usepackage{etoolbox}                           % Misc
\MakeOuterQuote{"}                              % For fancy quotes
% Turn off URLs in citations. Don't delete this in case we need it again.
% https://tex.stackexchange.com/questions/425392/how-remove-links-from-apacite
%\renewenvironment{APACrefURL}[1][]{}{}
%\AtBeginEnvironment{APACrefURL}{\renewcommand{\url}[1]{}}

\title{Israel IQP Project Proposal}
\author{Evan Goldstein, Valeria Kopper, Christopher Myers, Zachary Zlotnick}
\date{November 2018}

\begin{document}
\pagenumbering{roman}
\maketitle

\renewcommand\abstractname{Summary} % "Abstract" -> "Summary", the header lies
\begin{abstract}
This short section at the top should be an executive summary of the rest of the document, something like an abstract. It should be very short (one paragraph at most) and extremely concise, covering the proposal at a very high level. It may be appropriate to put sponsor acknowledgments here if they're needed. \textbf{\textit{This section might not be needed!}}
\end{abstract}

\tableofcontents
\newpage
\pagenumbering{arabic}

% Actual content begins here!
\doublespacing

\section{Introduction}
\textit{Not part of this submission.}

\newpage
\section{Background}
\subsection{Electric vehicles, the smart grid, \& vehicle-to-grid charging}

Electric vehicles (EVs) are an energy efficient alternative to traditional combustion-powered vehicles, relying on built-in batteries and motors to store energy and move the vehicle. Traditional vehicles produce CO$^2$ emissions from burning gasoline, but convert only 10-15\% of the energy in the gasoline into energy to move the vehicle. In comparison, a hybrid vehicle --- combining combustion and electric technology --- can yield 30-40\% efficiency \cite{Zhu2015DistributedGrid}, with the remaining energy similarly wasted. Pure EVs operate on much simpler principles and are much more efficient.

Modern EVs are currently limited in both battery capacity and range, presenting a usage-awareness challenge to consumers who wish to switch. On the low end of the scale, cars can have as little as 18.8 kWh in energy storage with a range of only 100 miles, while higher end models can store up to 53 kWh for a range of 220 miles \cite{Ustun2015ImpactSystems}. Depending on travel habits and local availability of charging stations, this may limit adoption. EVs also increasingly require dedicated, high-power charging systems, namely "fast chargers" requiring up to 70 kW to charge one vehicle \cite{Ustun2015ImpactSystems}. If many of these vehicles were in use in one area, the load on the local electric grid during peak hours could be enormous.

To help satisfy energy requirements and reduce carbon emissions, a number of renewable energy options are available to Eilat and Israel as a whole, most notably solar power. Solar power requires large amounts of exposure to the sun and huge swathes of land to generate power effectively (modern photovoltaic solar panels are only 10-17\% efficient \cite{Zhu2015DistributedGrid}), but solar panels can also be mounted as rooftop installations. In a best-case scenario with solar panels on every roof, up to 32\% of Israel's energy requirements could be satisfied by solar, while in a more realistic scenario of solar panels on only large roofs ($>$800m$^2$) up to 7\% could be satisfied \cite{Vardimon2011AssessmentIsrael}, and this is only for rooftop deployments --- far more power could be generated by dedicated solar farms, which Israelis also support \cite{Dipersio2018PhotovoltaicAcceptance}. Unfortunately, the volatile nature of solar energy (dependent on weather and time of day) can cause problems with satisfying power demand \cite{Lu2015IntroductionPEVs} and deployment of dedicated solar farms would likely require significant upgrades to the existing power grid \cite{Vardimon2011AssessmentIsrael}.

To solve many electric demand problems, such as how to accommodate the rise in demand towards the evenings (peak time) the concept of a \textit{smart grid} has been proposed. Briefly, a smart grid is a power grid where power is regulated intelligently by computers in an on-demand fashion, usually incorporating "microgrids" that contain a number of "loads" (things that draw power, like a house) and local sources of power, such as batteries or rooftop solar panels \cite{Lu2015IntroductionPEVs}. One proposed use of the smart grid is to connect EVs when they're not in use, charging their batteries when power is more readily available and discharging during peak time \cite{Mahmud2015PowerEV}. Ideally, this would help relieve stress on power generators and reduce waste from long-range power transmission \cite{Zhu2015DistributedGrid}. Together the EVs would serve as a kind of distributed battery; these concepts are the core of vehicle-to-grid (V2G) charging.

Unfortunately, the technology needed has not been fully investigated and the transition phase may be difficult. Consumers may be reluctant to switch to EVs due to their range or availability of charging stations \cite{Zhu2015DistributedGrid}, which is a circular problem since fewer consumers transitioning implies fewer charging stations. Universal standards for charging cars and the complicated control circuity involved in two-way power flow may also need to be developed \cite{Ustun2015ImpactSystems, Yeshayahou2011IsraelCharging}. A smart grid and V2G charging in Eilat may need adapting to their local environment, depending on factors such as Israeli travel habits (a car can only serve as a battery for the grid if it's not in use), frequency of use, or Eilat citizens' willingness to switch to EVs.

\subsection{Existing economy (Zach)}
Information about the current makeup of Israel's automotive industry (if they have one) and peripheral industries such as car repair or manufacturing. A discussion of the usage of the costs of oil imports might also be relevant.

\begin{itemize}
\item includes all the activities related to the manufacture of light- and heavy-weight vehicles.
\item repair and maintenance services
\end{itemize}

\cite{}

\subsection{Transportation (Evan)}
This should be the social effects situation, but since we haven't conducted any research on what consumers in Israel think of electric cars, we can't call it that. Basically this should cover the travel habits of modern Israeli citizens, esp. in Eilat. For example, how often do they drive? What's the average length of a trip? How much does the average person spend on gas?

\subsection{Oil imports \& Israeli independence (Valeria)}

A discussion of the extremely complicated relations between Israel and its neighbors could be useful. We should discuss where Israel gets its oil from in particular, and what kind of effect that has on their relations with other countries.

\subsection{Summary}

\newpage
\section{Methods}
\textit{Not part of this submission.}

% Bibliography -- automatically managed in APA style, on a new page
\newpage
\bibliographystyle{apacite}
\bibliography{refs}

\end{document}

\documentclass{report}                         % Make it a report 
\usepackage{indentfirst}                       % All paragraphs indented
\usepackage{setspace}                           % For 1.5 spacing
\usepackage[hidelinks]{hyperref}      % Make table of contents clickable
\usepackage[margin=1in]{geometry} % 1 inch margin
\usepackage[english]{babel}                     % For "fancy" quotes
\usepackage[autostyle, english=american]{csquotes} % Same
\usepackage{apacite}                            % Use APA citations
\usepackage{etoolbox}                           % Misc
\usepackage{titlesec}
\titleformat{\chapter}[display]{\normalfont\bfseries}{}{0pt}{\Huge} % Make headers reasonable again 
\titlespacing*{\chapter}{0pt}{-30pt}{0pt}
\titlespacing*{\section}{0pt}{0pt}{0pt}
\titlespacing*{\subsection}{0pt}{0pt}{-5pt}
\MakeOuterQuote{"}                              % For fancy quotes


\title{Israel IQP Project Proposal}
\author{Evan Goldstein, Valeria Kopper, Christopher Myers, Zachary Zlotnick}
\date{November 2018}

\begin{document}
\pagenumbering{roman}
\maketitle

\renewcommand\abstractname{Summary} % "Abstract" -> "Summary", the header lies

\tableofcontents
\newpage
\pagenumbering{arabic}

% Actual content begins here!
\doublespacing

\chapter{Introduction}

Modern Israel relies on conventional (gasoline and diesel-powered) vehicles for transportation. With nearly 70\% of Israeli citizens driving to work and another 18\% taking buses, commute times have increased 30\% over the past decade \cite{Dori2018IsraeliRoads}, increasing oil consumption and worsening the dependence on it. The environmental impacts of oil usage are well known, but in Israel's case oil usage leads to unique diplomatic problems, with over 99\% of its oil imported from its Arab neighbors \cite{Engber2006WhereOil}. This presents a serious motivation for reducing oil dependence and replacing it with solar power.

To that end, Eilat is investigating the potential of switching vehicles to electric vehicles (EVs), importantly with a vehicle-to-grid (V2G) charging system.  In a V2G system, EVs connected to a charger would have two-way power flow, charging during times of excess power in the grid and discharging when demand is higher. This presents several challenges, such as required upgrades to the power grid, increasing the number of charging facilities, and increasing solar power production.

To help regulate power sources and consumption on both localized and large scales, some cities have begun implementing smart grids as part of various "smart city" initiatives, complete with vehicle-to-grid charging. The premise of smart grids is to regulate power at a local level using networked communications to determine what power goes where, and where it comes from. For instance, V2G systems might kick in and start charging cars when power demand is low, or roof-mounted solar panels might deliver power into the grid if it's detected that the house they're mounted on is using less power than is being generated. However, these systems are currently under development \cite{2018Nuvve...} and have not been fully implemented.

The problem with existing proposals is that they do not take into account Eilat's demographics and power sources. Eilat aims to transition to electric vehicles \textit{and} increase solar usage, which can interfere with part of the smart-grid/V2G benefits -- charging during times of decreased load, which tend to be at night when solar panels cannot generate power. Without proper power management, it's entirely possible that an EV owner might wake up one morning only to find that their car has discharged overnight so much that their car no longer has the range to make it to work. The applicability of V2G charging to Eilat has also not been investigated, with little to no available data on whether Eilat citizens would accept the switch to electric cars. Significant upgrades to Eilat's power grid would be required \cite{Vardimon2011AssessmentIsrael}, but whether this is feasible or not is undetermined. 

The goal of this project is to assess the technological viability and requirements of Eilat's goal to have 90\% electric vehicles by 2040, and develop a roadmap or guide for its implementation in order to aid the city of Eilat in switching to electric vehicles.
We aim to collect data on social acceptance of EVs in Eilat, investigate the current state of Eilat's power grid, and analyze social acceptance \& technological data to determine whether implementation by 2040 is realistic. Although adapting the technology to Eilat's situation and convincing citizens to convert to EVs may prove challenging, it opens unique opportunities. Eilat's effort can prove V2G's potential, demonstrate a working EV-powered city, and act as a role model for other cities looking to switch.

\newpage
\chapter{Background}
\textit{Need an introduction here!}

\section{Transportation}
Eilat is a relatively small city with a population of 47,800. Within the city center, you can walk almost everywhere. There is only one bus company, Egged. \cite{TransportationEilat} "Get Taxi", an Uber-like service, is also available. 

In March 2018, a report published by the Bank of Israel found that commute times in the country have increased by 30\% in the last decade. This is likely because private and commercial vehicles account for how 69\% of Israelis get to work nationwide. Following that, 18\% take a bus. Tel Aviv and Haifa are very close to the national numbers, but in Jerusalem, the statistic changes drastically, to 54\% car and 32\% bus \cite{Dori2018IsraeliRoads}. By comparison, in European cities like Barcelona and Brussels public transportation accounted for 40\% of all commuting and in Paris, 70\%. The report found that those who do take the bus to work do so not because it is convenient or efficient, but because they cannot afford to own a car, and therefore have no choice. The report came after the International Monetary Fund said Israel’s transportation infrastructure was insufficient and was threatening to be a drag on economic growth. 

Israeli car ownership has increased rapidly in the past 20 years. In 1995, car ownership was climbing at 6-7\% per year --- among the fastest in the world at the time \cite{Slater1995IsraelCulture}. A 2012 report from the Israel Tax Authority showed that car owners in the center and north of the country drove the least --- around 14,000 kilometers a year on average between the Tel Aviv, Jerusalem and Haifa areas --- while car owners in far-north Metula area drove the most, at 22,000 kilometers a year. Eilat car owners, who are the farthest from the country’s center, drove the second most at 19,000 kilometers a year \cite{Schmil2012WhatHow}. The report states that between 1995 and 2011, tax revenue from cars more than doubled. Additionally, whether due to increasing green-awareness or Israel’s Green Tax Reform, cars on Israel's roads are becoming less damaging to the environment\cite{Schmil2012WhatHow}.

Luxury car ownership has also increased rapidly, doubling in recent years \cite{Weinglass2018Rev-upIsrael}. Economist Yaron Zalicha told the Times of Israel: “The low interest rates of recent years allowed car purchasers to borrow more money and incentivized consumption over saving. …In addition, the government recently made it easier for people to import cars individually, which incentivizes the import of expensive cars” \cite{Weinglass2018Rev-upIsrael}. Some see the rise of luxury cars as being a direct result of a growing economy. As in many places, there is a stigma against owning a luxury car. It is a common stereotype in Israel that the owners of luxury cars are criminals or tax evaders. This is partly because until a new law goes into effect in 2019 it is legal in Israel to purchase apartments and cars using cash.

Israel does not have a particularly thriving car culture scene. This is largely limited by the government. Speed limits are low, and every single modification to a vehicle, even changing the wheel size, requires special permission. New car purchases are taxed at 83\%, making the purchase of a car very expensive.

\section{Electric vehicles \& solar power}
\subsection{Electric vehicles}
Electric vehicles (EVs) are an energy efficient alternative to traditional combustion-powered vehicles, relying on built-in batteries and motors to store energy and move the vehicle. Traditional vehicles produce carbon dioxide (CO$^2$) emissions from burning gasoline, but convert only 10-15\% of the energy in the gasoline into energy to move the vehicle. In comparison, a hybrid vehicle --- combining combustion and electric technology --- can yield 30-40\% efficiency \cite{Zhu2015DistributedGrid}, with the remaining energy similarly wasted. Pure EVs operate on much simpler principles, saving energy and reducing carbon emissions.

Modern EVs are currently limited in both battery capacity and range, presenting a usage-awareness challenge to consumers who wish to switch. On the low end of the scale, cars can have as little as 18.8 kWh in energy storage with a range of only 100 miles, while higher end models can store up to 53 kWh for a range of 220 miles \cite{Ustun2015ImpactSystems} --- for reference, Israelis use 6,893 kWh per capita as of 2018 \cite{2018Key2018}. Depending on travel habits and local availability of charging stations, this may limit adoption. EVs also increasingly require dedicated, high-power charging systems, namely "fast chargers" requiring up to 70 kW (reference: A normal laptop might draw 60-70 W while charging) to charge one vehicle in 30 minutes \cite{Ustun2015ImpactSystems}. If many of these chargers were in use in one area, the load on the local electric grid during peak hours could be enormous.

\subsection{Solar energy}
To help satisfy energy requirements and reduce carbon emissions, a number of renewable energy options are available to Eilat and Israel as a whole, most notably solar power. Solar power requires large amounts of exposure to the sun and huge swathes of land to generate power effectively. Modern photovoltaic solar panels are only 10-17\% efficient \cite{Zhu2015DistributedGrid}), generating 150-200 watts per square meter, so supplying Israel with its average demand of 6.7 gigawatts would require 33-45 km$^2$ of contiguous solar panels.

Unfortunately, the volatile nature of solar energy (affected by weather but entirely dependent on time of day, solar panels cannot supply power during the night) can cause problems with satisfying power demand \cite{Lu2015IntroductionPEVs} and the deployment of dedicated solar farms would likely require significant upgrades to the existing power grid \cite{Vardimon2011AssessmentIsrael} in both capacity and technology. Combined with the high energy demands imposed by a fleet of electric vehicles, a desperately-needed increase in solar power and 90\% EVs (as per Eilat's goal) could be expected to require significant changes to Eilat's electrical grid.

\subsection{V2G \& smart grids}
To address both these problems and demand problems like accommodating the rise in demand towards the evenings (peak time), engineers have proposed the concept of a \textit{smart grid}. Briefly, a smart grid is a power grid where power production, storage, and transmission is regulated intelligently by computers in an on-demand fashion. These usually involve "microgrids" that contain a number of "loads" (something that draws power, like a house) and local sources of power, such as batteries or rooftop solar panels \cite{Lu2015IntroductionPEVs}. One proposed use of smart grids is to connect EVs when they're not in use, charging their batteries when power is more readily available and discharging during peak time \cite{Mahmud2015PowerEV} or times of exceptional demand. Ideally, this would help relieve stress on power generators and reduce waste from long-range power transmission \cite{Zhu2015DistributedGrid}. Together the connected EVs would serve as a distributed battery; these concepts are the core of vehicle-to-grid (V2G) charging. It's important to understand that these concepts are inseparable --- a V2G charging station \textit{is} a smart grid component, implementing V2G without needed upgrades to the grid would be ineffective or difficult to manage.

Unfortunately, the technology needed has not been fully investigated and the transition phase may be difficult. Consumers may be reluctant to switch to EVs due to their range or availability of charging stations \cite{Zhu2015DistributedGrid}. This may be a cyclical problem since fewer consumers transitioning implies fewer charging stations, since there would be little incentive to build many stations for a small fleet of EVs. Universal standards for charging cars and the complicated control circuity involved in two-way power flow may also need to be developed \cite{Ustun2015ImpactSystems, Yeshayahou2011IsraelCharging}, not to mention the actual infrastructure and control systems underpinning smart grids. A smart grid and V2G charging in Eilat may also need adapting to their local environment, depending on factors such as Israeli travel habits (a car can only serve as a battery for the grid if it's not in use, for instance), frequency of use, or Eilat citizens' individual willingness and financial readiness to switch to EVs.

\section{Existing automotive industry}
Israel is not traditionally considered a strong player within the global automotive industry; however, they are proving to be a country on the rise in this regard. As a nation, Israel is known for the environment it has provided for technology start-ups to thrive. That in mind, many global brands are deciding to invest in the "high-end automotive technology sector." Some of these global brands include Intel, looking to apply its software automation in Jerusalem; Volkswagen AG, trying to popularize the ride hailing service called "Gett,"; even BMW AG, who made a large investment in a transit app called "Moovit" \cite{Coutinho2018IsraelIndustry}. 

With regard to the traditional facets of the automotive industry, the Israeli market also has companies specializing in Tier 1, 2, and Tier 3 supply; Tier 1 supplying companies with components for final assembly as well as aftermarket spare parts, Tier 2 supplying Tier 1 with sub-assembly components, and Tier 3 supplying simple engineered materials for Tier 2 suppliers, respectively. As of 2015, Israel has over 60 companies working in Tier 1 aftermarket supply creating spare parts and accessories such as batteries, ventilation, gears and valves. Israel also has over 50 companies that supply sub-assembly components to original equipment manufacturers (OEMs) \cite{MinistryofEconomyandIndustryStateofIsraelTheIsrael}. These companies may begin to shift over time as Israel seeks alternatives to traditional, inefficient cars, and moves toward lighter, smarter, and technologically savvy electric vehicles.

Keeping in mind that Israel is looking to move toward smart cars and efficient vehicles, there are a number of notable trends occurring in the world right now. One shift being made is the reduction of vehicle weight by using alternative materials, given that a car that is 10\% lighter would use 6-8\% less fuel \cite{MinistryofEconomyandIndustryStateofIsraelTheIsrael}. Other popular trends lie in the information technology sector, enabling vehicles to communicate with other vehicles and become more automated. These technologies include driver assistance systems, automated and smart transportation, and information security. All of these technologies aim to reduce the inherent error that human drivers have, decreasing commute times, reducing the impact of potential crashes, and keeping personal information out of the hands of hackers \cite{MinistryofEconomyandIndustryStateofIsraelTheIsrael}. Directly related to this increase in technology within the automotive sector, there exist some ethical considerations that must be taken into account. While the idea of more efficient vehicles would not receive much opposition, technology which takes control from the driver is heavily debated. Questions regarding the culpability of the driver in certain situations come into play, as well as driver rights, manufacturer liability, and implications for future generations.

Responding to this fundamental shift in the way the world is viewing transportation, Israel has responded with a few ideas of their own. For example, the trending navigation app "Waze" was based in Israel before they were bought out by Google in 2013. Waze encourages communication between drivers via a mobile app (not while driving obviously) where users can report traffic jams, vehicles stopped on the shoulder, hazards, and even police officers. Another Israeli company, "Cellint," monitors all active cell phones and detects their exact location on the roads every 30 seconds, collecting data to use for traffic management, research, and time-to-destination analysis. Other Israeli companies such as "Argus" aim to address the issue of information security, ensuring that data can be shared between vehicles seamlessly without interference from hackers \cite{MinistryofEconomyandIndustryStateofIsraelTheIsrael}. The list of Israel-based companies and their various applications goes on, however their goals all relate to creating safer, smoother and more efficient journeys for drivers.

Overall, the Israeli automotive industry is a developing one. This is not to say that it is underdeveloped, but that it has focused its lens on emerging technologies and brand new global needs. International companies have focused their attention to Israel in search of new viable automobile-related technologies. In addition to the cutting-edge technologies that have sprouted in Israel, they also have made efforts to vertically integrate many of them - ensuring that there is support from initial design to final production \cite{MinistryofEconomyandIndustryStateofIsraelTheIsrael}. Israel's location enables them to get the best of both commercial worlds; they can easily trade with Asia and Europe.


\section{Oil imports \& Israeli independence}
The modern world has a developed a deep reliance on oil, and the State of Israel is no exception. Oil is required by many daily activities. While many have attempted to decrease this strong dependence, such as through investment in renewable energy production, there has been little success. For some countries this means large oil production and exports, many times being the driving force of their economy. For other countries, this implies being tied and subject to the countries responsible for sourcing their oil imports. Among the former category fall many middle eastern nations. Israel, however, produces incredibly small quantities of oil, as little as only a couple thousand barrels of oil a day \cite{Engber2006WhereOil}, forcing the nation to heavily rely on oil imports. Such a small oil production does not allow Israel to sustain more than 1\% of its oil needs, which means it imports over 99\% of its oil \cite{Engber2006WhereOil}. Many of these imports originate from its neighboring countries; countries with whom Israel has a long and complicated history of tense relationships. This generates an undesired and convoluted vulnerability for Israel.
    
Israel's oil import distribution among exporting countries is undisclosed because, as Israel's minister of national infrastructures (in 2004) stated, "Israel's situation is complicated. We don't have diplomatic relations with most of the countries from which we import oil." \cite{Engber2006WhereOil} Over the years Israel has imported oil from a wide range of countries, including but not limited to, Angola, Colombia, Mexico, Egypt, and Norway \cite{Engber2006WhereOil}. More recently these imports have shifted towards a large majority originating from Russia and Kazakhstan \cite{Engber2006WhereOil}. 
	
Since its establishment, the State of Israel has continuously sought oil resources closer to home in order to increase stability and safety as well as decreasing import costs. This notion rose to particular prominence during Israel's 1973 oil crisis, which led to large imports from Iran through a pipe referred to as the "TIPline" \cite{Engber2006WhereOil}. This pipe allowed for Iranian oil extraction from the Red Sea; however, this only lasted until the Shah was overthrown in 1979 \cite{Engber2006WhereOil}. This led to Israel seizing control of oil sources in the Sinai Fields in Egypt after the Six-Day War took place \cite{Engber2006WhereOil}.  Afterwards Israel and Egypt have maintained a friendly relationship within the oil context; however, Israel's demand for oil has grown, leaving Egypt as a minor source of import. Its overall growth and increase of oil demand has further fostered Israel's need to decrease its dependence on other nations for almost the entirety of their oil resources. 

Ideally, Israel can decrease its dependence on other nations by decreasing its oil consumption. Decreasing energy consumption should be a global goal, but at this point in time the focus should on decreasing oil consumption specifically. On the other hand, focusing on the development of energy consumption in future years progressing towards the use of renewable energy instead of oil consumption would yield great benefits in terms of environmental, social and political aspects. This can be achieved by focusing on renewable energy production and shifting towards electric machines, such as electric vehicles, versus oil-run machines. Israel has the opportunity to expand upon its untapped potential for renewable energy production, especially solar polar, which would grant Israel an increased independence from its oil imports and undesired ties to neighboring countries. 

\section{Summary}

Overall, Israel stands to gain much by decreasing its oil reliance, increasing solar power production, and switching to electric vehicles. In the process it would gain a more reliable power system, greener vehicles, and reduced political difficulties involved with importing oil from its less-than-friendly neighbors. However, the process of making the switch over the next 20 years can be expected to be complicated, with numerous technological developments required, in-depth study of the social effects and social feasibility of electric vehicles, and possibly mitigation to avoid threatening the Israeli automotive industry. Additional research and investigation is required to determine the feasibility and implications of Eilat's plan to switch to 90\% electric vehicles by 2040.\

\newpage
\chapter{Methods}
\textit{Not part of this submission.}

% Bibliography -- automatically managed in APA style, on a new page
\newpage
\bigskip
\bibliographystyle{apacite}
\bibliography{refs}

\end{document}
